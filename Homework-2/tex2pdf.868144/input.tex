\documentclass[]{article}
\usepackage{lmodern}
\usepackage{amssymb,amsmath}
\usepackage{ifxetex,ifluatex}
\usepackage{fixltx2e} % provides \textsubscript
\ifnum 0\ifxetex 1\fi\ifluatex 1\fi=0 % if pdftex
  \usepackage[T1]{fontenc}
  \usepackage[utf8]{inputenc}
\else % if luatex or xelatex
  \ifxetex
    \usepackage{mathspec}
  \else
    \usepackage{fontspec}
  \fi
  \defaultfontfeatures{Ligatures=TeX,Scale=MatchLowercase}
\fi
% use upquote if available, for straight quotes in verbatim environments
\IfFileExists{upquote.sty}{\usepackage{upquote}}{}
% use microtype if available
\IfFileExists{microtype.sty}{%
\usepackage{microtype}
\UseMicrotypeSet[protrusion]{basicmath} % disable protrusion for tt fonts
}{}
\usepackage[margin=1in]{geometry}
\usepackage{hyperref}
\hypersetup{unicode=true,
            pdftitle={ENVE 681 HW 2},
            pdfauthor={Nicola Horscroft},
            pdfborder={0 0 0},
            breaklinks=true}
\urlstyle{same}  % don't use monospace font for urls
\usepackage{color}
\usepackage{fancyvrb}
\newcommand{\VerbBar}{|}
\newcommand{\VERB}{\Verb[commandchars=\\\{\}]}
\DefineVerbatimEnvironment{Highlighting}{Verbatim}{commandchars=\\\{\}}
% Add ',fontsize=\small' for more characters per line
\usepackage{framed}
\definecolor{shadecolor}{RGB}{248,248,248}
\newenvironment{Shaded}{\begin{snugshade}}{\end{snugshade}}
\newcommand{\KeywordTok}[1]{\textcolor[rgb]{0.13,0.29,0.53}{\textbf{#1}}}
\newcommand{\DataTypeTok}[1]{\textcolor[rgb]{0.13,0.29,0.53}{#1}}
\newcommand{\DecValTok}[1]{\textcolor[rgb]{0.00,0.00,0.81}{#1}}
\newcommand{\BaseNTok}[1]{\textcolor[rgb]{0.00,0.00,0.81}{#1}}
\newcommand{\FloatTok}[1]{\textcolor[rgb]{0.00,0.00,0.81}{#1}}
\newcommand{\ConstantTok}[1]{\textcolor[rgb]{0.00,0.00,0.00}{#1}}
\newcommand{\CharTok}[1]{\textcolor[rgb]{0.31,0.60,0.02}{#1}}
\newcommand{\SpecialCharTok}[1]{\textcolor[rgb]{0.00,0.00,0.00}{#1}}
\newcommand{\StringTok}[1]{\textcolor[rgb]{0.31,0.60,0.02}{#1}}
\newcommand{\VerbatimStringTok}[1]{\textcolor[rgb]{0.31,0.60,0.02}{#1}}
\newcommand{\SpecialStringTok}[1]{\textcolor[rgb]{0.31,0.60,0.02}{#1}}
\newcommand{\ImportTok}[1]{#1}
\newcommand{\CommentTok}[1]{\textcolor[rgb]{0.56,0.35,0.01}{\textit{#1}}}
\newcommand{\DocumentationTok}[1]{\textcolor[rgb]{0.56,0.35,0.01}{\textbf{\textit{#1}}}}
\newcommand{\AnnotationTok}[1]{\textcolor[rgb]{0.56,0.35,0.01}{\textbf{\textit{#1}}}}
\newcommand{\CommentVarTok}[1]{\textcolor[rgb]{0.56,0.35,0.01}{\textbf{\textit{#1}}}}
\newcommand{\OtherTok}[1]{\textcolor[rgb]{0.56,0.35,0.01}{#1}}
\newcommand{\FunctionTok}[1]{\textcolor[rgb]{0.00,0.00,0.00}{#1}}
\newcommand{\VariableTok}[1]{\textcolor[rgb]{0.00,0.00,0.00}{#1}}
\newcommand{\ControlFlowTok}[1]{\textcolor[rgb]{0.13,0.29,0.53}{\textbf{#1}}}
\newcommand{\OperatorTok}[1]{\textcolor[rgb]{0.81,0.36,0.00}{\textbf{#1}}}
\newcommand{\BuiltInTok}[1]{#1}
\newcommand{\ExtensionTok}[1]{#1}
\newcommand{\PreprocessorTok}[1]{\textcolor[rgb]{0.56,0.35,0.01}{\textit{#1}}}
\newcommand{\AttributeTok}[1]{\textcolor[rgb]{0.77,0.63,0.00}{#1}}
\newcommand{\RegionMarkerTok}[1]{#1}
\newcommand{\InformationTok}[1]{\textcolor[rgb]{0.56,0.35,0.01}{\textbf{\textit{#1}}}}
\newcommand{\WarningTok}[1]{\textcolor[rgb]{0.56,0.35,0.01}{\textbf{\textit{#1}}}}
\newcommand{\AlertTok}[1]{\textcolor[rgb]{0.94,0.16,0.16}{#1}}
\newcommand{\ErrorTok}[1]{\textcolor[rgb]{0.64,0.00,0.00}{\textbf{#1}}}
\newcommand{\NormalTok}[1]{#1}
\usepackage{graphicx,grffile}
\makeatletter
\def\maxwidth{\ifdim\Gin@nat@width>\linewidth\linewidth\else\Gin@nat@width\fi}
\def\maxheight{\ifdim\Gin@nat@height>\textheight\textheight\else\Gin@nat@height\fi}
\makeatother
% Scale images if necessary, so that they will not overflow the page
% margins by default, and it is still possible to overwrite the defaults
% using explicit options in \includegraphics[width, height, ...]{}
\setkeys{Gin}{width=\maxwidth,height=\maxheight,keepaspectratio}
\IfFileExists{parskip.sty}{%
\usepackage{parskip}
}{% else
\setlength{\parindent}{0pt}
\setlength{\parskip}{6pt plus 2pt minus 1pt}
}
\setlength{\emergencystretch}{3em}  % prevent overfull lines
\providecommand{\tightlist}{%
  \setlength{\itemsep}{0pt}\setlength{\parskip}{0pt}}
\setcounter{secnumdepth}{0}
% Redefines (sub)paragraphs to behave more like sections
\ifx\paragraph\undefined\else
\let\oldparagraph\paragraph
\renewcommand{\paragraph}[1]{\oldparagraph{#1}\mbox{}}
\fi
\ifx\subparagraph\undefined\else
\let\oldsubparagraph\subparagraph
\renewcommand{\subparagraph}[1]{\oldsubparagraph{#1}\mbox{}}
\fi

%%% Use protect on footnotes to avoid problems with footnotes in titles
\let\rmarkdownfootnote\footnote%
\def\footnote{\protect\rmarkdownfootnote}

%%% Change title format to be more compact
\usepackage{titling}

% Create subtitle command for use in maketitle
\newcommand{\subtitle}[1]{
  \posttitle{
    \begin{center}\large#1\end{center}
    }
}

\setlength{\droptitle}{-2em}
  \title{ENVE 681 HW 2}
  \pretitle{\vspace{\droptitle}\centering\huge}
  \posttitle{\par}
  \author{Nicola Horscroft}
  \preauthor{\centering\large\emph}
  \postauthor{\par}
  \predate{\centering\large\emph}
  \postdate{\par}
  \date{October 17, 2017}


\begin{document}
\maketitle

This problem is designed to introduce work with R and to do some basic
data manipulation and simple programming. You will download and use
precipitation data (see ppt.xlsx file, given in inches) for the
Philadelphia Airport. The rows represent years from 1980-2014 and the
columns represent months from January to December.

\begin{enumerate}
\def\labelenumi{\arabic{enumi}.}
\tightlist
\item
  Read the data into R
\end{enumerate}

\begin{Shaded}
\begin{Highlighting}[]
\KeywordTok{library}\NormalTok{(readxl)}
\KeywordTok{library}\NormalTok{(tidyverse)}
\end{Highlighting}
\end{Shaded}

\begin{verbatim}
## Loading tidyverse: ggplot2
## Loading tidyverse: tibble
## Loading tidyverse: tidyr
## Loading tidyverse: readr
## Loading tidyverse: purrr
## Loading tidyverse: dplyr
\end{verbatim}

\begin{verbatim}
## Warning: package 'dplyr' was built under R version 3.4.2
\end{verbatim}

\begin{verbatim}
## Conflicts with tidy packages ----------------------------------------------
\end{verbatim}

\begin{verbatim}
## filter(): dplyr, stats
## lag():    dplyr, stats
\end{verbatim}

\begin{Shaded}
\begin{Highlighting}[]
\KeywordTok{library}\NormalTok{(lubridate)}
\end{Highlighting}
\end{Shaded}

\begin{verbatim}
## 
## Attaching package: 'lubridate'
\end{verbatim}

\begin{verbatim}
## The following object is masked from 'package:base':
## 
##     date
\end{verbatim}

\begin{Shaded}
\begin{Highlighting}[]
\CommentTok{# setwd("F:/ENVE 681")}
\NormalTok{ppt <-}\StringTok{ }\KeywordTok{read_excel}\NormalTok{(}\StringTok{"HW 2/ppt.xlsx"}\NormalTok{)}
\CommentTok{#View(ppt)}

\NormalTok{xppt <-}\StringTok{ }\NormalTok{ppt }\OperatorTok
\StringTok{  }\KeywordTok{mutate}\NormalTok{(}\DataTypeTok{year =} \DecValTok{1980}\OperatorTok{:}\DecValTok{2014}\NormalTok{)}
\end{Highlighting}
\end{Shaded}

\begin{enumerate}
\def\labelenumi{\arabic{enumi}.}
\setcounter{enumi}{1}
\tightlist
\item
  Calculate the total annual precipitation for each year and plot these
  data versus year
\end{enumerate}

\begin{Shaded}
\begin{Highlighting}[]
\NormalTok{answer_}\DecValTok{2}\NormalTok{ <-}\StringTok{ }\NormalTok{xppt }\OperatorTok\StringTok{  }
\StringTok{  }\KeywordTok{gather}\NormalTok{(}\DataTypeTok{key =} \StringTok{"month"}\NormalTok{, }\DataTypeTok{value =} \StringTok{"value"}\NormalTok{, Jan}\OperatorTok{:}\NormalTok{Dec) }\OperatorTok
\StringTok{  }\KeywordTok{group_by}\NormalTok{(year) }\OperatorTok\StringTok{ }
\StringTok{  }\KeywordTok{summarize}\NormalTok{(}\DataTypeTok{sum =} \KeywordTok{sum}\NormalTok{(value))}

\KeywordTok{ggplot}\NormalTok{(answer_}\DecValTok{2}\NormalTok{, }\KeywordTok{aes}\NormalTok{(year, sum)) }\OperatorTok{+}
\StringTok{  }\KeywordTok{geom_line}\NormalTok{() }\OperatorTok{+}
\StringTok{  }\KeywordTok{scale_x_continuous}\NormalTok{(}\DataTypeTok{breaks =} \KeywordTok{seq}\NormalTok{(}\DecValTok{1980}\NormalTok{, }\DecValTok{2014}\NormalTok{, }\DecValTok{2}\NormalTok{)) }\OperatorTok{+}
\StringTok{  }\KeywordTok{theme_bw}\NormalTok{() }\OperatorTok{+}\StringTok{ }
\StringTok{  }\KeywordTok{theme}\NormalTok{(}\DataTypeTok{axis.text.x =} \KeywordTok{element_text}\NormalTok{(}\DataTypeTok{angle =} \DecValTok{90}\NormalTok{, }\DataTypeTok{hjust =} \DecValTok{1}\NormalTok{, }\DataTypeTok{vjust =} \FloatTok{0.5}\NormalTok{),}
        \DataTypeTok{panel.grid.minor.x =} \KeywordTok{element_blank}\NormalTok{()) }\OperatorTok{+}\StringTok{ }
\StringTok{  }\KeywordTok{labs}\NormalTok{(}\DataTypeTok{x =} \StringTok{"Year"}\NormalTok{, }\DataTypeTok{y =} \StringTok{"Precipitation}\CharTok{\textbackslash{}n}\StringTok{(in)"}\NormalTok{, }\DataTypeTok{title =} \StringTok{"Total Annual Precipitation"}\NormalTok{)}
\end{Highlighting}
\end{Shaded}

\includegraphics{ENVE_681_HW_2_files/figure-latex/unnamed-chunk-2-1.pdf}

\begin{enumerate}
\def\labelenumi{\arabic{enumi}.}
\setcounter{enumi}{2}
\tightlist
\item
  For a given year (you can pick), calculate and display the minimum,
  maximum, and mean monthly precipitation
\end{enumerate}

\begin{Shaded}
\begin{Highlighting}[]
\NormalTok{xppt }\OperatorTok
\StringTok{  }\KeywordTok{gather}\NormalTok{(}\DataTypeTok{key =} \StringTok{"month"}\NormalTok{, }\DataTypeTok{value =} \StringTok{"value"}\NormalTok{, Jan}\OperatorTok{:}\NormalTok{Dec) }\OperatorTok\StringTok{ }
\StringTok{  }\KeywordTok{filter}\NormalTok{(year }\OperatorTok{==}\StringTok{ }\DecValTok{1990}\NormalTok{) }\OperatorTok\StringTok{ }
\StringTok{  }\KeywordTok{group_by}\NormalTok{(year) }\OperatorTok\StringTok{ }
\StringTok{  }\KeywordTok{summarize}\NormalTok{(}\DataTypeTok{mean =} \KeywordTok{mean}\NormalTok{(value), }
            \DataTypeTok{min =} \KeywordTok{min}\NormalTok{(value),}
            \DataTypeTok{max =} \KeywordTok{max}\NormalTok{(value))}
\end{Highlighting}
\end{Shaded}

\begin{verbatim}
## # A tibble: 1 x 4
##    year   mean   min   max
##   <int>  <dbl> <dbl> <dbl>
## 1  1990 2.9825  1.17  6.08
\end{verbatim}

\begin{enumerate}
\def\labelenumi{\arabic{enumi}.}
\setcounter{enumi}{3}
\tightlist
\item
  Calculate the mean monthly precipitation for each month and plot the
  values using a bar chart. Now add ``error bars'' showing the standard
  deviations of the monthly means.
\end{enumerate}

\begin{Shaded}
\begin{Highlighting}[]
\NormalTok{answer_}\DecValTok{4}\NormalTok{ <-}\StringTok{ }\NormalTok{xppt }\OperatorTok\StringTok{ }
\StringTok{  }\KeywordTok{gather}\NormalTok{(}\DataTypeTok{key =} \StringTok{"month"}\NormalTok{, }\DataTypeTok{value =} \StringTok{"value"}\NormalTok{, Jan}\OperatorTok{:}\NormalTok{Dec) }\OperatorTok\StringTok{ }
\StringTok{  }\KeywordTok{mutate}\NormalTok{(}\DataTypeTok{month =} \KeywordTok{factor}\NormalTok{(month, }\DataTypeTok{levels =} \KeywordTok{c}\NormalTok{(}\StringTok{"Jan"}\NormalTok{, }\StringTok{"Feb"}\NormalTok{, }\StringTok{"Mar"}\NormalTok{, }\StringTok{"April"}\NormalTok{, }\StringTok{"May"}\NormalTok{, }\StringTok{"June"}\NormalTok{,}
                                          \StringTok{"July"}\NormalTok{, }\StringTok{"Aug"}\NormalTok{, }\StringTok{"Sept"}\NormalTok{, }\StringTok{"Oct"}\NormalTok{, }\StringTok{"Nov"}\NormalTok{, }\StringTok{"Dec"}\NormalTok{))) }\OperatorTok\StringTok{ }
\StringTok{  }\KeywordTok{group_by}\NormalTok{(month) }\OperatorTok\StringTok{ }
\StringTok{  }\KeywordTok{summarize}\NormalTok{(}\DataTypeTok{mean =} \KeywordTok{mean}\NormalTok{(value),}
            \DataTypeTok{std =} \KeywordTok{sd}\NormalTok{(value))}

\KeywordTok{ggplot}\NormalTok{(answer_}\DecValTok{4}\NormalTok{, }\KeywordTok{aes}\NormalTok{(month, mean)) }\OperatorTok{+}\StringTok{ }
\StringTok{  }\KeywordTok{geom_col}\NormalTok{(}\DataTypeTok{fill =} \StringTok{"#9C9C9C"}\NormalTok{) }\OperatorTok{+}\StringTok{ }
\StringTok{  }\KeywordTok{geom_errorbar}\NormalTok{(}\KeywordTok{aes}\NormalTok{(}\DataTypeTok{x =}\NormalTok{ month, }\DataTypeTok{ymin =}\NormalTok{ mean }\OperatorTok{-}\StringTok{ }\NormalTok{std, }\DataTypeTok{ymax =}\NormalTok{ mean }\OperatorTok{+}\StringTok{ }\NormalTok{std), }\DataTypeTok{color =} \StringTok{"black"}\NormalTok{) }\OperatorTok{+}\StringTok{ }
\StringTok{  }\KeywordTok{theme_bw}\NormalTok{() }\OperatorTok{+}\StringTok{ }
\StringTok{  }\KeywordTok{labs}\NormalTok{(}\DataTypeTok{x =} \StringTok{"Month"}\NormalTok{, }\DataTypeTok{y =} \StringTok{"Average Precipitation}\CharTok{\textbackslash{}n}\StringTok{(in)"}\NormalTok{, }
       \DataTypeTok{title =} \StringTok{"Average Precipitation by Month"}\NormalTok{, }\DataTypeTok{subtitle =} \StringTok{"Data from 1980 to 2014"}\NormalTok{)}
\end{Highlighting}
\end{Shaded}

\includegraphics{ENVE_681_HW_2_files/figure-latex/unnamed-chunk-4-1.pdf}

\begin{enumerate}
\def\labelenumi{\arabic{enumi}.}
\setcounter{enumi}{4}
\tightlist
\item
  Plot all the monthly precipitation values consecutively as a line plot
\end{enumerate}

\begin{Shaded}
\begin{Highlighting}[]
\NormalTok{answer_}\DecValTok{5}\NormalTok{ <-}\StringTok{ }\NormalTok{xppt }\OperatorTok\StringTok{ }
\StringTok{  }\KeywordTok{gather}\NormalTok{(}\DataTypeTok{key =} \StringTok{"month"}\NormalTok{, }\DataTypeTok{value =} \StringTok{"value"}\NormalTok{, Jan}\OperatorTok{:}\NormalTok{Dec) }\OperatorTok\StringTok{ }
\StringTok{  }\KeywordTok{mutate}\NormalTok{(}\DataTypeTok{month =} \KeywordTok{factor}\NormalTok{(month, }\DataTypeTok{levels =} \KeywordTok{c}\NormalTok{(}\StringTok{"Jan"}\NormalTok{, }\StringTok{"Feb"}\NormalTok{, }\StringTok{"Mar"}\NormalTok{, }\StringTok{"April"}\NormalTok{, }\StringTok{"May"}\NormalTok{, }\StringTok{"June"}\NormalTok{,}
                                          \StringTok{"July"}\NormalTok{, }\StringTok{"Aug"}\NormalTok{, }\StringTok{"Sept"}\NormalTok{, }\StringTok{"Oct"}\NormalTok{, }\StringTok{"Nov"}\NormalTok{, }\StringTok{"Dec"}\NormalTok{))) }\OperatorTok
\StringTok{  }\KeywordTok{mutate}\NormalTok{(}\DataTypeTok{date =} \KeywordTok{ifelse}\NormalTok{(}\KeywordTok{as.numeric}\NormalTok{(month) }\OperatorTok{<}\StringTok{ }\DecValTok{9}\NormalTok{, }
                       \KeywordTok{paste}\NormalTok{(}\KeywordTok{as.character}\NormalTok{(year), }\StringTok{"-0"}\NormalTok{, }\KeywordTok{as.numeric}\NormalTok{(month), }\StringTok{"-01"}\NormalTok{, }\DataTypeTok{sep =} \StringTok{""}\NormalTok{), }
                       \KeywordTok{paste}\NormalTok{(}\KeywordTok{as.character}\NormalTok{(year), }\StringTok{"-"}\NormalTok{, }\KeywordTok{as.numeric}\NormalTok{(month), }\StringTok{"-01"}\NormalTok{, }\DataTypeTok{sep =} \StringTok{""}\NormalTok{))) }\OperatorTok\StringTok{ }
\StringTok{  }\KeywordTok{mutate}\NormalTok{(}\DataTypeTok{date =} \KeywordTok{as.POSIXct}\NormalTok{(}\KeywordTok{strptime}\NormalTok{(date, }\DataTypeTok{format =} \StringTok{"%Y-%m-%d"}\NormalTok{)))}
  
\KeywordTok{ggplot}\NormalTok{(answer_}\DecValTok{5}\NormalTok{, }\KeywordTok{aes}\NormalTok{(date, value)) }\OperatorTok{+}\StringTok{ }
\StringTok{  }\KeywordTok{geom_line}\NormalTok{() }\OperatorTok{+}\StringTok{ }
\StringTok{  }\KeywordTok{scale_x_datetime}\NormalTok{(}\DataTypeTok{date_breaks =} \StringTok{"5 years"}\NormalTok{, }\DataTypeTok{date_labels =} \StringTok{"%Y"}\NormalTok{) }\OperatorTok{+}\StringTok{ }
\StringTok{  }\KeywordTok{theme_bw}\NormalTok{() }\OperatorTok{+}\StringTok{ }
\StringTok{  }\KeywordTok{labs}\NormalTok{(}\DataTypeTok{x =} \StringTok{"Date"}\NormalTok{, }\DataTypeTok{y =} \StringTok{"Precipitation}\CharTok{\textbackslash{}n}\StringTok{(in)"}\NormalTok{, }\DataTypeTok{title =} \StringTok{"Yearly Precipitation"}\NormalTok{, }\DataTypeTok{subtitle =} \StringTok{"Data from 1980 to 2014"}\NormalTok{)}
\end{Highlighting}
\end{Shaded}

\includegraphics{ENVE_681_HW_2_files/figure-latex/unnamed-chunk-5-1.pdf}


\end{document}
